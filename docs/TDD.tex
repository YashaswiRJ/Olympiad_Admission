\documentclass[12pt,a4paper]{article}
\usepackage[utf8]{inputenc}
\usepackage{graphicx}
\usepackage{hyperref}
\usepackage{listings}
\usepackage{color}
\usepackage{tikz}
\usepackage{float}
\usepackage{enumitem}

\definecolor{codegreen}{rgb}{0,0.6,0}
\definecolor{codegray}{rgb}{0.5,0.5,0.5}
\definecolor{codepurple}{rgb}{0.58,0,0.82}

\lstdefinestyle{mystyle}{
    commentstyle=\color{codegreen},
    keywordstyle=\color{magenta},
    stringstyle=\color{codepurple},
    basicstyle=\ttfamily\footnotesize,
    breakatwhitespace=false,
    breaklines=true,
    keepspaces=true,
    showspaces=false,
    showstringspaces=false,
    showtabs=false,
    tabsize=2
}

\lstset{style=mystyle}

\title{Technical Design Document\\
       Olympiad Admission System}
\author{IIT Kanpur}
\date{\today}

\begin{document}
\maketitle
\tableofcontents
\newpage

\section{System Architecture}
\subsection{Overview}
The Olympiad Admission System is a web application with a React frontend and a Python backend. The frontend communicates with the backend via RESTful APIs for validation and ranking. Data is stored in browser localStorage for session continuity.

\subsection{Component Hierarchy}
\begin{verbatim}
App
├── Sidebar
└── Main Content
    ├── Dashboard
    ├── UploadCSV
    ├── ValidationPreference
    ├── RankingPage
    ├── SeatAllocation
    └── UploadSeatMatrix
\end{verbatim}

\section{Component Specifications}

\subsection{App Component}
\begin{lstlisting}[language=JavaScript]
// App.js
function App() {
  return (
    <Router>
      <div className="app">
        <Sidebar />
        <div className="main-content">
          <Routes>
            <Route path="/" element={<Dashboard />} />
            <Route path="/upload" element={<UploadCSV />} />
            <Route path="/validate" element={<ValidationPreference />} />
            <Route path="/ranking" element={<RankingPage />} />
            <Route path="/generate-seat-allocation" element={<SeatAllocation />} />
          </Routes>
        </div>
      </div>
    </Router>
  );
}
\end{lstlisting}

\subsection{Sidebar Component}
\begin{lstlisting}[language=JavaScript]
// Sidebar.js
const Sidebar = () => {
  const [isCollapsed, setIsCollapsed] = useState(false);
  
  return (
    <div className={`sidebar ${isCollapsed ? 'collapsed' : ''}`}>
      {/* Navigation items */}
    </div>
  );
};
\end{lstlisting}

\subsection{UploadCSV Component}
\begin{itemize}
    \item Handles CSV upload, parsing, and validation
    \item Displays data in a paginated, searchable table
    \item Allows row removal and data saving to localStorage
\end{itemize}

\subsection{ValidationPreference Component}
\begin{itemize}
    \item Validates student preferences via backend API
    \item Displays validation results
    \item Stores validation data in localStorage
\end{itemize}

\subsection{RankingPage Component}
\begin{itemize}
    \item Fetches and displays rankings from backend API
    \item Uses validation data from localStorage
    \item Provides search and pagination
\end{itemize}

\subsection{SeatAllocation and UploadSeatMatrix}
\begin{itemize}
    \item Upload and manage seat matrix
    \item Generate and review seat allocations
\end{itemize}

\section{Data Flow}
\begin{enumerate}
    \item CSV Upload → Validation (backend) → Store validation data in localStorage
    \item Ranking Generation (backend) using validation data
    \item Seat Allocation using rankings and seat matrix
\end{enumerate}

\section{Storage Implementation}
\begin{lstlisting}[language=JavaScript]
// Save validation data
localStorage.setItem('validationData', JSON.stringify(validationResult));
// Retrieve validation data
const validationData = JSON.parse(localStorage.getItem('validationData'));
\end{lstlisting}

\section{UI/UX Implementation}
\begin{itemize}
    \item Responsive design for all modules
    \item Paginated and searchable tables
    \item Clear error and success notifications
    \item Accessibility for keyboard navigation
\end{itemize}

\section{Error Handling}
\begin{itemize}
    \item File upload and CSV validation errors
    \item Backend/API errors
    \item Data persistence and recovery
\end{itemize}

\section{Testing Strategy}
\begin{itemize}
    \item Unit tests for CSV parsing and validation
    \item Integration tests for API communication
    \item UI tests for all major workflows
\end{itemize}

\section{Deployment}
\begin{itemize}
    \item Build with npm run build
    \item Deploy static files to web server
    \item Ensure backend API is accessible
\end{itemize}

\end{document} 